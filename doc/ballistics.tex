\documentclass{article}

\usepackage{amsmath}
\usepackage{tikz}

\begin{document}

\title{Mathematics of \texttt{Ballistics} class}
\date{}
\maketitle

\section{Line-line intersection}

(Based on page 304 of book "Graphic Gems")

Suppose we have two lines $P$ and $Q$,
represented by the points $(p_1, p_2)$ and $(q_1, q_2)$.
Define the vectors $p$ and $q$ by
\begin{align*}
    p &= p_2 - p_1; \\
    q &= q_2 - q_1.
\end{align*}
Then every point in the line $P$ will be of the form $tp + p_1$
for some real number $t$.
Similarly, every point in the line $Q$ will be of the form $sq + q_1$.
They intersect when
\begin{align*}
    sq + q_1 &= tp + p_1 \\
    sq + q_1 - p_1 &= tp
\end{align*}
Taking the cross product of both sides with $q$ gives
\begin{align*}
    q \times (sq + q_1 - p_1) &= t p \times q \\
    s q \times q + q \times (q_1 - p_1) &= t p \times q
\end{align*}
Since $q \times q = 0$, this simplifies to
\begin{equation*}
    q \times (q_1 - p_1) = t p \times q.
\end{equation*}
Now, take the dot product of both sides with $p \times q$;
this will enable us to isolate $t$:
\begin{align*}
    \langle q \times (q_1 - p_1), p \times q \rangle
        &= t \langle p \times q, p \times q \rangle \\
    \langle q \times (q_1 - p_1), p \times q \rangle &= t ||p \times q|| \\
    t &= \frac{\langle q \times(q_1 - p_1), p \times q \rangle}{||p \times q||}.
\end{align*}
Since we have $p_1$, $q_1$, $q$ and $p$, we can compute $t$;
so, the intersection is
\begin{equation*}
    tp + p_1,
\end{equation*}
provided the value $|| p \times q ||$ is non-zero.

\section{Angle of projection}
\begin{figure}[h]
    \centering
    \begin{tikzpicture}[scale = 2]
        \begin{scope}[->, gray]
            \draw (0, 0, 0) -- (0, 0, 2);
            \draw (0, 0, 0) -- (0, 2, 0);
            \draw (0, 0, 0) -- (2, 0, 0);
            \path (0, 0, 2.5) node {$z$};
            \path (0, 2.5, 0) node {$y$};
            \path (2.5, 0, 0) node {$x$};
        \end{scope}

        \begin{scope}[rotate around y = -90]
            \draw (0, 0) -- (1.75, 1.75);
            \draw (1, 0) arc (0:45:1);
            \path (1.5, 0.5) node {$\alpha$};
            \path (2, 2) node {$A$};
            \draw[gray, dotted] (0, 1.5) -- (1.5, 1.5) -- (1.5, 0);
        \end{scope}

        \begin{scope}[rotate around y = -45]
            \draw (0, 0) -- (1.75, 1.75);
            \draw (1, 0) arc (0:45:1);
            \path (1.5, 0.5) node {$\alpha$};
            \path (2, 2) node {$B$};
            \draw[gray, dotted] (0, 1.5) -- (1.5, 1.5) -- (1.5, 0);
        \end{scope}

        \begin{scope}[rotate around x = 90]
            \draw (0, 0) -- (1.25, 1.25);
            \draw (0, 1) arc (90:45:1);
            \path (0.5, 1.5) node {$\beta$};
            \path (1.5, 1.5) node {$W$};
            \draw[gray, dotted] (0, 1) -- (1, 1) -- (1, 0);
        \end{scope}

        \begin{scope}[rotate around y = 45, rotate around x = 45]
            \draw (0, 1.5) arc (45:90:1.5);
            \path (-0.5, 2.25) node {$\gamma$};
        \end{scope}
    \end{tikzpicture}
    \caption{Problem of angle of projection}
    \label{fig:angle-of-projection}
\end{figure}

This problem is depicted in the figure \ref{fig:angle-of-projection}.
Given the angle $\gamma$ between the vectors $A$ and $B$
and the angle $\beta$ of their projection in a plane,
determine the angle $\alpha$ that both make with the plane.

We can assume, without loss of generality,
that $||A|| = ||B|| = 1$,
and that $A$ is in the plane $yz$.
The coordinates of the vector $A$ is
\begin{equation*}
    A = (0, \sin \alpha, \cos \alpha).
\end{equation*}
The vector $W$, the projection of $B$ in the plane $xz$,
is a multiple of
\begin{equation*}
    W' = (\sin \beta, 0, \cos \beta).
\end{equation*}
In the plane $yW$, the vector $B$ can be written as
\begin{equation*}
    B = (\cos \alpha, \sin \alpha).
\end{equation*}
As this is in the plane $yW$, in the tridimensional space,
this is
\begin{align*}
    B &= \cos \alpha W' + \sin \alpha y \\
      &= \cos \alpha (\sin \beta, 0, \cos \beta) + \sin \alpha (0, 1, 0) \\
      &= (\cos \alpha \sin \beta, \sin \alpha, \cos \alpha \cos \beta).
\end{align*}

As $||A|| = ||B|| = 1$, the inner product of $A$ and $B$ gives $\cos \gamma$.
Thas is,
\begin{align*}
    \cos \gamma &= \langle A, B \rangle \\
                &= \langle (0, \sin \alpha, \cos \alpha),
            (\cos \alpha \sin \beta, \sin \alpha, \cos \alpha \cos \beta) \rangle \\
                &= \sin^2 \alpha + \cos^2 \alpha \cos \beta
\end{align*}
Now, replace $\sin^2 \alpha$ by $1 - \cos^2 \alpha$, giving
\begin{align*}
    \cos \gamma &= 1 - \cos^2 \alpha + \cos^2 \alpha \cos \beta \\
    \cos \gamma - 1 &= \cos^2 \alpha( \cos \beta - 1 ) \\
    \frac{\cos \gamma - 1}{\cos \beta - 1} &= \cos^2 \alpha \\
    \cos \alpha &= \sqrt{\frac{\cos \gamma - 1}{\cos \beta - 1}} \\
    \alpha &= \arccos \sqrt{\frac{\cos \gamma - 1}{\cos \beta - 1}}
\end{align*}

\section{Finding the axis}

Using the same figure (\ref{fig:angle-of-projection}),
now we seek a way to compute the vector $y$
(the rotation axis)
using only $A$, $B$ and the newly-discovered $\alpha$.

The key observation is that the rotation axis $C$,
the vector $M = (A + B)/2$ and the cross-product $A \times B$
all lie on the same plane
(since the angle between any of them and $A$ is the same angle as
themselves and $B$).
Thus, we need just to rotate $M$ around $M \times (A \times B)$
by some angle to get $C$.
We will compute this angle $\delta$ now.

Let's impose $||C|| = 1$.
We know then that
\begin{align*}
    ||M|| \cos \delta &= \langle C, M \rangle \\
                      &= \langle C, (A + B) / 2 \rangle \\
                      &= \frac{ \langle C, A \rangle + \langle C, B \rangle }{2}
\end{align*}
Now, as $\langle C, A \rangle$ gives $\cos( \pi/2 - \alpha )$
(and similarly for $B$),
we have
\begin{align*}
    ||M|| \cos \delta &= \frac{ \cos(\pi/2 - \alpha) + \cos(\pi/2 - \alpha)}{2} \\
    \cos \delta &= \frac{\cos(\pi/2 - \alpha)}{||M||}.
\end{align*}

\section{Computing the angles}

Suppose now we already computed the following four vectors:
\begin{itemize}
    \item $u = \texttt{up}$,
        the vector that points ``up'' along the main rotation axis
    \item $f = \texttt{front}$,
        the vector that points to the ``front'' of the laser
    \item $c = \texttt{center}$,
        a point in the space which represents the center of the laser
    \item $l = \texttt{left}$,
        a vector that points ``left'' along the secondary rotation axis
\end{itemize}

Vectors $u$ and $l$ are always orthogonal;
vectors $l$ and $f$ are always orthogonal;
vectors $u$ and $f$ are almost always not orthogonal.

We will now show how to update $f$ and $l$,
given a new measurement $p$ in the space,
in order to make the laser point to $p$.

First, replace $p$ by $p - c$,
thus moving the origin of the coordinate system to $c$.
(This was the whole reason to adding $c$ to our object;
now, we may assume that $c$ is the origin and ignore it altogether.)

The rotation around the \texttt{left} axis
will necessarily be in a plane orthogonal to $l$.
Since the plane passes through $u$
(as we are only moving $u$ up or down),
we can project $p$ in this plane
by projecting it in the axis $u$ and $u \times l$,
and adding these parts.

Call the projection $p_s$.
To compute the actual angle of rotation,
first compute the angle $\alpha$ between the projection and \texttt{front},
then rotate \texttt{front} around \texttt{left} by both $\alpha$ and $-\alpha$
and see what is closer to $p_s$;
the ``winning'' is the angle of rotation.

The computation around the \texttt{up} axis is trickier,
because neither \texttt{front} nor the rotated \texttt{front}
will be on the rotation plane.
Thus, we must project both $p$ and $f$ in the rotation plane
(spanned by $l$ and $u \times l$)
to compute the angle between them.

\end{document}
