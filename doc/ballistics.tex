\documentclass{article}

\usepackage{amsmath}
\usepackage{tikz}

\begin{document}

\title{Mathematics of \texttt{Ballistics} class}
\date{}
\maketitle

\section{Line-line intersection}

(Based on page 304 of book "Graphic Gems")

Suppose we have two lines $P$ and $Q$,
represented by the points $(p_1, p_2)$ and $(q_1, q_2)$.
Define the vectors $p$ and $q$ by
\begin{align*}
    p &= p_2 - p_1; \\
    q &= q_2 - q_1.
\end{align*}
Then every point in the line $P$ will be of the form $tp + p_1$
for some real number $t$.
Similarly, every point in the line $Q$ will be of the form $sq + q_1$.
They intersect when
\begin{align*}
    sq + q_1 &= tp + p_1 \\
    sq + q_1 - p_1 &= tp
\end{align*}
Taking the cross product of both sides with $q$ gives
\begin{align*}
    q \times (sq + q_1 - p_1) &= t p \times q \\
    s q \times q + q \times (q_1 - p_1) &= t p \times q
\end{align*}
Since $q \times q = 0$, this simplifies to
\begin{equation*}
    q \times (q_1 - p_1) = t p \times q.
\end{equation*}
Now, take the dot product of both sides with $p \times q$;
this will enable us to isolate $t$:
\begin{align*}
    \langle q \times (q_1 - p_1), p \times q \rangle
        &= t \langle p \times q, p \times q \rangle \\
    \langle q \times (q_1 - p_1), p \times q \rangle &= t ||p \times q|| \\
    t &= \frac{\langle q \times(q_1, p_1), p \times q \rangle}{||p \times q||}.
\end{align*}
Since we have $p_1$, $q_1$, $q$ and $p$, we can compute $t$;
so, the intersection is
\begin{equation*}
    tp + p_1,
\end{equation*}
provided the value $|| p \times q ||$ is non-zero.
\end{document}
